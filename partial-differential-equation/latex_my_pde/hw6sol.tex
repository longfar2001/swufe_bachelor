\NeedsTeXFormat{LaTeX2e}% LaTeX 2.09 can't be used (nor non-LaTeX)
[1994/12/01]% LaTeX date must December 1994 or later
\documentclass[6pt]{article}
\pagestyle{headings}
\setlength{\textwidth}{18cm}
\setlength{\topmargin}{0in}
\setlength{\headsep}{0in}

\title{Introduction to PDEs, Fall 2022}
\author{\textbf{Homework 6} Solutions}
\date{}

\voffset -1.25cm \hoffset -3.5cm \textwidth 18cm \textheight 25cm
\renewcommand{\theequation}{\thesection.\arabic{equation}}
\renewcommand{\thefootnote}{\fnsymbol{footnote}}
\usepackage{amsmath}
\usepackage{amsthm}
%\usepackage{esint}
  \usepackage{paralist}
  \usepackage{graphics} %% add this and next lines if pictures should be in esp format
  \usepackage{epsfig} %For pictures: screened artwork should be set up with an 85 or 100 line screen
\usepackage{graphicx}
\usepackage{caption}
\usepackage{subcaption}
\usepackage{epstopdf}%This is to transfer .eps figure to .pdf figure; please compile your paper using PDFLeTex or PDFTeXify.
 \usepackage[colorlinks=true]{hyperref}
 \usepackage{multirow}
\input{amssym.tex}
\def\N{{\Bbb N}}
\def\Z{{\Bbb Z}}
\def\Q{{\Bbb Q}}
\def\R{{\Bbb R}}
\def\C{{\Bbb C}}
\def\SS{{\Bbb S}}

\newtheorem{theorem}{Theorem}[section]
\newtheorem{corollary}{Corollary}
%\newtheorem*{main}{Main Theorem}
\newtheorem{lemma}[theorem]{Lemma}
\newtheorem{proposition}{Proposition}
\newtheorem{conjecture}{Conjecture}
\newtheorem{solution}{Solution}
%\newtheorem{proof}{Proof}
 \numberwithin{equation}{section}
%\newtheorem*{problem}{Problem}
%\theoremstyle{definition}
%\newtheorem{definition}[theorem]{Definition}
\newtheorem{remark}{Remark}
%\newtheorem*{notation}{Notation}
\newcommand{\ep}{\varepsilon}
\newcommand{\eps}[1]{{#1}_{\varepsilon}}
\newcommand{\keywords}

\def\bb{\begin}
\def\bc{\begin{center}}       \def\ec{\end{center}}
\def\ba{\begin{array}}        \def\ea{\end{array}}
\def\be{\begin{equation}}     \def\ee{\end{equation}}
\def\bea{\begin{eqnarray}}    \def\eea{\end{eqnarray}}
\def\beaa{\begin{eqnarray*}}  \def\eeaa{\end{eqnarray*}}
\def\hh{\!\!\!\!}             \def\EM{\hh &   &\hh}
\def\EQ{\hh & = & \hh}        \def\EE{\hh & \equiv & \hh}
\def\LE{\hh & \le & \hh}      \def\GE{\hh & \ge & \hh}
\def\LT{\hh & < & \hh}        \def\GT{\hh & > & \hh}
\def\NE{\hh & \ne & \hh}      \def\AND#1{\hh & #1 & \hh}

\def\r{\right}
\def\lf{\left}
\def\hs{\hspace{0.5cm}}
\def\dint{\displaystyle\int}
\def\dlim{\displaystyle\lim}
\def\dsup{\displaystyle\sup}
\def\dmin{\displaystyle\min}
\def\dmax{\displaystyle\max}
\def\dinf{\displaystyle\inf}

\def\al{\alpha}               \def\bt{\beta}
\def\ep{\varepsilon}
\def\la{\lambda}              \def\vp{\varphi}
\def\da{\delta}               \def\th{\theta}
\def\vth{\vartheta}           \def\nn{\nonumber}
\def\oo{\infty}
\def\dd{\cdots}               \def\pa{\partial}
\def\q{\quad}                 \def\qq{\qquad}
\def\dx{{\dot x}}             \def\ddx{{\ddot x}}
\def\f{\frac}                 \def\fa{\forall\,}
\def\z{\left}                 \def\y{\right}
\def\w{\omega}                \def\bs{\backslash}
\def\ga{\gamma}               \def\si{\sigma}
\def\iint{\int\!\!\!\!\int}
\def\dfrac#1#2{\frac{\displaystyle {#1}}{\displaystyle {#2}}}
\def\mathbb{\Bbb}
\def\bl{\Bigl}
\def\br{\Bigr}
\def\Real{\R}
\def\Proof{\noindent{\bf Proof}\quad}
\def\qed{\hfill$\square$\smallskip}

\begin{document}
\maketitle

\textbf{Name}:\rule{1 in}{0.001 in} \\


\begin{enumerate}
\item In measure theory, there are two additional convergence manners for $f_n\rightarrow f$: convergence in measure (called convergence in probability) and convergence almost everywhere (convergence almost surely).  This should give you a flavor that probability is a measure and vice versa.  Convergence in measure states that for each fixed $\varepsilon>0$
    \[m\big(\{x\in\Omega: |f_n(x)-f(x)|\geq\varepsilon\}\big)\rightarrow 0, \text{~as~}n\rightarrow \infty,\]
    and convergence almost everywhere means that the measure of the non-convergence region is zero, i.e., for each fixed $\varepsilon>0$
    \[m\big(\{x\in\Omega: \lim_{n\rightarrow\infty}|f_n(x)-f(x)|\geq\varepsilon\}\big)=0.\]
i) what are the relationships between these two convergence manners?  Prove your claims or give a counter-example.

ii) what are their relationships between strong convergence (convergence in $L^2$ for instance)?  Prove your claims or give a counter-example.

I would like to point out that convergence is global behavior in strong contrast to pointwise convergence since all the points are involved in the convergence limit.
\begin{solution}
There facts are from your undergraduate course, and I assume that each of you know these statements and their proofs.  I present them here to give you a taste how they apply to/in PDEs.
\end{solution}



\item  It is known that strong convergence implies weak convergence, while not the converse.  One counter-example we mentioned in class is $f_n(x):=\sin nx$ over $(0,2\pi)$.

(i)  Prove that $\sin nx \nrightarrow 0 $ in $L^2((0,2\pi))$.

(ii)  Prove that
$\sin nx \rightharpoonup 0$ weakly by showing
\[\int_0^{2\pi} g(x)\sin nx dx\rightarrow 0=\Big(\int_0^{2\pi} g(x)0  dx\Big),\forall g\in L^2((0,2\pi)).\]
If suffices even if $g\in L^1$.  Hint: Riemann--Lebesgue lemma.
\begin{solution}
(i).  It is very easy to show that $\Vert \sin nx\Vert_{L^2((0,2\pi))}\not \rightarrow 0$ by straightforward calculations, hence the strong convergence is impossible.  (What is the value?)

(ii).  Applying Riemann--Lebesgue lemma gives the desired limit.  I already presented partial approach in class and I assume/need that you know how to prove this lemma in detail.  One student, if you all remember, mentioned that you learnt this in Calculus.  It is possible but I still doubt it as the $L^2$-norm was not introduced by then.  Wish me wrong.
\end{solution}

\item  We recall that $f_n(x) \rightharpoonup f(x)$ weakly in $L^p$ (resp. convergence in distribution) if for any $\phi\in L^q$ (resp. continuous and bounded), which is its conjugate space with $\frac{1}{p}+\frac{1}{q}=1$, we have that
\[\int_\Omega f_n\phi dx\rightarrow \int_\Omega f\phi dx.\]
Here we see that for any $g$ in $L^q$
\[<\cdot, g>=\int_\Omega \cdot g\]
defines a bounded linear functional for $L^p$.  Then we also call $L^q$ \emph{the dual space} of $L^p$ since any element in $L^q$ defines a functional for $L^q$.

(i)  Another type of convergence that you may see sometimes is $\Vert f_n \Vert_p\rightarrow \Vert f\Vert_p$, which merely states the convergence of a sequence of real numbers.  Prove that if $f_n \rightarrow f$ in $L^p$, then $\Vert f_n \Vert_p\rightarrow \Vert f\Vert_p$ (Use Minkowski triangle inequality); however the opposite statement is not necessarily true.  Give a counter-example and show it;

(ii) We have proved that strong convergence in $L^p$ implies the weak convergence by Holder's inequality, however, the opposite statement is not necessarily true.  For example, prove that $\sin nx$ converges to zero weakly, but not strongly in $L^p$.  Hint: Riemann--Lebesgue lemma;

(iii) Prove that, if $f_n \rightharpoonup f$ weakly and $\Vert f_n \Vert_p\rightarrow \Vert f\Vert_p$, then $f_n\rightarrow f$ strongly.

\begin{solution}
Minkowski's triangle inequality states that, $\forall f,g\in L^p$, $p\in(1,\infty)$, we always have that $\Vert f+g\Vert_{L^p}\leq \Vert f\Vert_{L^p}+\Vert g\Vert_{L^p}$.  Then it is not hard to obtain that
\[\big| \Vert f_n\Vert_{L^p}-\Vert f\Vert_{L^p} \big|\leq \Vert f_n-f\Vert_{L^p}\rightarrow 0,\]
which proves the desired claim.  You can fill in the details yourself. I skip presenting a counter-example here, however, here is a hint on how you can construct such a counter-example:  think of $f_n$ and $f$ as points in a plane and their norms measure the distance from the origin, therefore $\Vert f_n-f\Vert_{L^p}\rightarrow 0$ means that the distance between $f_n$ and $f$ goes to zero, while $\Vert f_n\Vert_{L^p}\rightarrow \Vert f\Vert_{L^p}$ merely means that the distance of $f_n$ to the origin converges to that of $f$.  Now it is not hard to surmise that the former implies the latter, but not the other way.  I assume that you can find a counter-example.

Finally, we shall prove that though each condition in does not, while both conditions, imply the strong convergence.  To prove this, let us divide our discussions into the following cases:

case 1: $p=2$.  Then the conclusion is straightforward following Cauchy--Schwarz inequality.  I assume that you have no problem proving this case;

case 2: $p>2$.  We first see that for any $z\in \mathbb R$
\[|z+1|^p\geq c|z|^p+pz+1,\]
where $c$ is a positive constant independent of $z$.  (In order to prove this fact, we just need to show that $\frac{|z+1|^p-pz-1}{|z|^p}$ has a positive lower bounded $c$ over $\mathbb R$).  Now we can let $z=\frac{f_n-f}{f}$ in this inequality, multiply it by $|f|^p$ and then integrate the new one over $\Omega$ to obtain
\[\int_\Omega|f_n|^pdx\geq \int_\Omega |f|^pdx+p\int_\Omega |f|^{p-2}f(f_n-f)dx+c\int_\Omega |f_n-f|^pdx.\]
Since $f_n \rightharpoonup f$ weakly, we see that the second integral on the right hand size of the equality converges to zero (think of $|f|^{p-2}f$ as a test function).  On the other hand, we have that $\int_\Omega |f_n|^pdx\rightarrow \int_\Omega |f|^pdx$ thanks to the strong convergence, therefore we must have
\[\int_\Omega |f_n-f|^pdx\rightarrow 0,\]
which implies the strong convergence.

case 3: $p\in(1,2)$.  The proof of this part is a little bit tricky.  Similar as above, we can show (by straightforward calculations) that $\forall z\in\mathbb R$
\begin{align*}
  |z+1| \geq & c|z|^p+pz+1,\text{~if~}|z|\geq1, \\
  |z+1| \geq & c|z|^2+pz+1,\text{~if~}|z|\geq1.
\end{align*}
In order to apply these inequalities, we shall choose $z=\frac{f_n-f}{f}$.  Denote
\[\Omega_n:=\{x\in\Omega; |z|\geq1\},\]
then we can have by the same calculations as above that
\begin{align*}
  \int_\Omega |f_n|^pdx =&\int_{\Omega\backslash\Omega_n}|f_n|^pdx+\int_{\Omega_n}|f_n|^pdx   \\
   =&\int_{\Omega\backslash\Omega_n}|z+1|^p|f|^pdx+\int_{\Omega_n}|z+1|^p|f|^pdx   \\
   \geq & \int_{\Omega\backslash\Omega_n}(c|z|^2+pz+1)|f|^pdx+\int_{\Omega_n}(c|z|^p+pz+1)|f|^pdx,   \\
\end{align*}
which implies, in light of the formula of $z$, that
\[\int_{\Omega\backslash\Omega_n}(f_n-f)^2|f|^{p-2}dx+\int_{\Omega_n}|f_n-f|^pdx\rightarrow 0.\]
Both integrals are nonnegative, hence both should converge to zero
\[\int_{\Omega\backslash\Omega_n}(f_n-f)^2|f|^{p-2}dx\rightarrow 0, \int_{\Omega_n}|f_n-f|^pdx\rightarrow 0.\]
In particular, we only need to show that
\[\int_{\Omega\backslash\Omega_n}|f_n-f|^pdx\rightarrow 0.\]
For this purpose, we shall apply Holder's inequality or Schwarz's inequality as following.  Note that $|f_n-f|<|f|$ in $\Omega\backslash\Omega_n$, then we have that
\begin{align*}
  \int_{\Omega\backslash\Omega_n}|f_n-f|^pdx\leq  & \int_{\Omega\backslash\Omega_n}|f|^{p-1} |f_n-f|dx  \\
   \leq & \Big(\int_{\Omega\backslash\Omega_n}|f|^{p} \Big)^\frac{1}{2}\Big(\int_{\Omega\backslash\Omega_n}|f|^{p-2} |f_n-f|^2dx \Big)^\frac{1}{2} \\
  \leq  &   \Big(\int_{\Omega}|f|^{p} \Big)^\frac{1}{2}\Big(\int_{\Omega\backslash\Omega_n}|f|^{p-2} |f_n-f|^2dx \Big)^\frac{1}{2}\rightarrow 0, \\
\end{align*}
which is the desired claim and the proof completes.
\end{solution}

\end{enumerate}


\end{document}
\endinput




