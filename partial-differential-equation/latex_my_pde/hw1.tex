\NeedsTeXFormat{LaTeX2e}% LaTeX 2.09 can't be used (nor non-LaTeX)
[1994/12/01]% LaTeX date must December 1994 or later
\documentclass[6pt]{article}
\pagestyle{headings}
\setlength{\textwidth}{18cm}
\setlength{\topmargin}{0in}
\setlength{\headsep}{0in}

\title{Introduction to PDEs, Fall 2022}
\author{\textbf{Homework 1} Due Oct 8}
\date{}

\voffset -2cm \hoffset -1.5cm \textwidth 16cm \textheight 24cm
\renewcommand{\theequation}{\thesection.\arabic{equation}}
\renewcommand{\thefootnote}{\fnsymbol{footnote}}
\usepackage{amsmath}
\usepackage{amsthm}
%\usepackage{esint}
  \usepackage{paralist}
  \usepackage{graphics} %% add this and next lines if pictures should be in esp format
  \usepackage{epsfig} %For pictures: screened artwork should be set up with an 85 or 100 line screen
\usepackage{graphicx}
\usepackage{caption}
\usepackage{subcaption}
\usepackage{epstopdf}%This is to transfer .eps figure to .pdf figure; please compile your paper using PDFLeTex or PDFTeXify.
 \usepackage[colorlinks=true]{hyperref}
 \usepackage{multirow}
\input{amssym.tex}
\def\N{{\Bbb N}}
\def\Z{{\Bbb Z}}
\def\Q{{\Bbb Q}}
\def\R{{\Bbb R}}
\def\C{{\Bbb C}}
\def\SS{{\Bbb S}}

\newtheorem{theorem}{Theorem}[section]
\newtheorem{corollary}{Corollary}
%\newtheorem*{main}{Main Theorem}
\newtheorem{lemma}[theorem]{Lemma}
\newtheorem{proposition}{Proposition}
\newtheorem{conjecture}{Conjecture}
\newtheorem{solution}{Solution}
%\newtheorem{proof}{Proof}
 \numberwithin{equation}{section}
%\newtheorem*{problem}{Problem}
%\theoremstyle{definition}
%\newtheorem{definition}[theorem]{Definition}
\newtheorem{remark}{Remark}
%\newtheorem*{notation}{Notation}
\newcommand{\ep}{\varepsilon}
\newcommand{\eps}[1]{{#1}_{\varepsilon}}
\newcommand{\keywords}


\def\bb{\begin}
\def\bc{\begin{center}}       \def\ec{\end{center}}
\def\ba{\begin{array}}        \def\ea{\end{array}}
\def\be{\begin{equation}}     \def\ee{\end{equation}}
\def\bea{\begin{eqnarray}}    \def\eea{\end{eqnarray}}
\def\beaa{\begin{eqnarray*}}  \def\eeaa{\end{eqnarray*}}
\def\hh{\!\!\!\!}             \def\EM{\hh &   &\hh}
\def\EQ{\hh & = & \hh}        \def\EE{\hh & \equiv & \hh}
\def\LE{\hh & \le & \hh}      \def\GE{\hh & \ge & \hh}
\def\LT{\hh & < & \hh}        \def\GT{\hh & > & \hh}
\def\NE{\hh & \ne & \hh}      \def\AND#1{\hh & #1 & \hh}

\def\r{\right}
\def\lf{\left}
\def\hs{\hspace{0.5cm}}
\def\dint{\displaystyle\int}
\def\dlim{\displaystyle\lim}
\def\dsup{\displaystyle\sup}
\def\dmin{\displaystyle\min}
\def\dmax{\displaystyle\max}
\def\dinf{\displaystyle\inf}

\def\al{\alpha}               \def\bt{\beta}
\def\ep{\varepsilon}
\def\la{\lambda}              \def\vp{\varphi}
\def\da{\delta}               \def\th{\theta}
\def\vth{\vartheta}           \def\nn{\nonumber}
\def\oo{\infty}
\def\dd{\cdots}               \def\pa{\partial}
\def\q{\quad}                 \def\qq{\qquad}
\def\dx{{\dot x}}             \def\ddx{{\ddot x}}
\def\f{\frac}                 \def\fa{\forall\,}
\def\z{\left}                 \def\y{\right}
\def\w{\omega}                \def\bs{\backslash}
\def\ga{\gamma}               \def\si{\sigma}
\def\iint{\int\!\!\!\!\int}
\def\dfrac#1#2{\frac{\displaystyle {#1}}{\displaystyle {#2}}}
\def\mathbb{\Bbb}
\def\bl{\Bigl}
\def\br{\Bigr}
\def\Real{\R}
\def\Proof{\noindent{\bf Proof}\quad}
\def\qed{\hfill$\square$\smallskip}

\begin{document}
\maketitle

\textbf{Name}:\rule{1 in}{0.001 in} \\
\begin{enumerate}

\item Let us consider the discrete model of random walk and assume for simplicity of computation that $\Delta x=\Delta t=1$.  Then we know that the discrete equation over the whole lattice is
\begin{equation}\label{discrete}
u(x,t+1)=\frac{1}{2}\big(u(x-1,t)+u(x+1,t)\big), x=\pm 1,\pm2, \pm3,..., t=0,1,2,...
\end{equation}
Suppose that initially, we put a number of $(4-x^2)^+$ particles at location $x$, where ``$+$" here denotes the positive part.  That is, the initial data are given such that $u(x,0)=4-x^2$ for $|x|\leq2$ and $u(x,0)\equiv0$ for $|x|\geq2$.

Let us consider $u(\pm3,t)$ at time $t=2$ for example.  It is easy to see that we need the values of $u(\pm4,1)$, which further require the values of $u(\pm 5,0)$.  Similarly, one traces back to $u(\pm6,0)$ in order to evaluate $u(\pm3,3)$.  Some of you may have recognized the mechanism/scheme as a binomial tree, I wish this gives you some intuition that the particles, which move locally to their neighboring sites at the next time step, will eventually spread out the whole region (we will see later in this course that the speed is infinite if $\Delta x\rightarrow 0^+$).  Or you can imagine that $u$ is the number of infected living organisms and the disease will eventually dominate the whole space if it spreads out randomly.

(i) plot $u(x,t)$ for $x=\pm3,\pm2,\pm1,0$ at time $t=0,1,...,6$, and connect the neighbouring dots with straight lines;

(ii) now set $\Delta x=\Delta t=0.5$ and plot $u(x,t)$ for $x=\pm3,\pm 2.5,\pm2,\pm 1.5,\pm1,0$ at time $t=0,1,...,6$. I suggest you use MATLAB or other computational software for the calculations.  Now you see, the discrete problem is, to be frank, simple but computationally annoying; do the same for $\Delta x=\Delta t=0.01$.  It becomes tedious with your bare hands but not if using a computer program.  1) That is why it makes well sense to study the continuous case, which approximates the discrete ones when $\Delta x$ and $\Delta t$ are small; 2) now you are learning the finite difference method solving the heat equation without realizing it;

(iii) now let us go back to the same problem with $\Delta x=\Delta t=1$ and the same initial condition $u(x,0)=(4-x^2)^+$ but now over the finite interval $(-5,5)$.  We are well set for $u(x,1)$ at all $x$ except $x=\pm 5$, because $x=\pm6$ is not considered in this finite interval.  Therefore, we need to set specific conditions for $u(\pm5,t)$ for any time $t$ in order to calculate $u(\pm4,t+1)$, so on and so forth.  This condition is called the boundary condition and it must be set for any PDE over a finite interval.  One of the types is the so-called Dirichlet boundary condition (you have seen this already) and we set that $u(\pm5,t)=0$ (or some other constant) for any $t>0$.  Suppose that $u(\pm5,t)=0$ for any $t\geq0$.  plot $u(x,t)$ for $x=\pm5,...,\pm1,0$ at time $t=0,1,...,6$, and connect the neighbouring dots with straight lines;

(iv)  do the same as in (iii) with $u(\pm 5,t)=2$ for any $t>0$;

(v)  do the same as in (iii) with $u(-5,t)=1$ and $u(5,t)=3$ for any $t>0$.  Now you can see that different boundary conditions can have different effects on the solution's behavior;

(vi)  do the same as in (iii) with $u(-5,t)=1$ and $u(5,t)=3$ for any $t>0$, but $u_0(x)=(x^2-4)^+$.  Now you can see that different initial conditions can have different effects on the solution's behavior;

\textbf{Note}: you can always, for your entertainment but no need to show me, choose $\Delta x$ and $\Delta t$ to another size, say $10^{-3}$, to see how the discrete solution approaches the continuum solution;

\item Now let us go back to our baby example in 1D lattice/grid: each particle at $x$ either moves to $x-\Delta x$ or $x+\Delta x$ with probability $\frac{1}{2}$, at any time $t$.  As I mentioned in class, some students may have concern that it is unfair or unrealistic to assume that each particle only moves to $x\pm \Delta x$ at the next time step, and it is possible that, for example, the particle may move to $x\pm\Delta x$ with probability $\frac{1}{4}$, and move to $x\pm2\Delta x$ with a small probability, say, $\frac{1}{8}$, and move to $x\pm3\Delta x$ with an even smaller probability, say, $\frac{1}{16}$, and to $x\pm 4\Delta x$, $x\pm 5\Delta x$...so on and so forth, with all the probabilities adding up to 1.  I do not have much to disagree with this possibility, however, this problem is designed to show that this case also leads to the classical heat equation, with merely a variation of the diffusion rate.  For the simplicity of our mathematical analysis, and without losing our generality, let us assume that each particle can only move to $x\pm\Delta x$, $x\pm 2\Delta x$, with probability
\[p(x\rightarrow x\pm\Delta x,t)=\frac{\alpha}{2}, p(x\rightarrow x\pm2\Delta x,t)=\frac{1-\alpha}{2},\]
hence $p(x\rightarrow x\pm3\Delta x,t)=p(x\rightarrow x\pm4\Delta x,t)=...=0$.   Derive the PDE for $u(x,t)$ by the microscopic approach.

\item Suppose that $u(x,t)$ moves to $x\pm\Delta x$  at any time $t$ with a probability $p$ which depends location $x$ but not time, i.e.,
\[p(x\rightarrow x\pm\Delta x,t)=\rho(x)\]
Put $D=\frac{\Delta x^2}{\Delta t}$ as $\Delta t \rightarrow 0^+$.  Derive the PDE for $u(x,t)$.  What if the probability also depends on time, i.e.,
\[p(x\rightarrow x\pm\Delta x,t)=\rho(x,t)\]
or
\[p(x\rightarrow x\pm\Delta x,t)=\rho(x,t+\Delta t).\]


\item  Now change the probability above as arrival--dependent, i.e.,
\[p(x\rightarrow x\pm\Delta x,t)=\rho(x\pm\Delta x).\]
Derive the PDE for $u(x,t)$.  Is the PDE the same as the one above?  Compare them and state your observation.

\item Let us consider a similar scenario in a 2D lattice with mesh size $\Delta x=\Delta y$.  Let $u(x,y,t)$ be the number of particles at location $(x,y)\in \mathbb{R}^2$ and time $t>0$.  Suppose that each particle, at the next time $t+\Delta$, moves northwards, southwards, westwards or eastwards with probability $\frac{1}{4}$, i.e., $p((x,y)\rightarrow (x\pm\Delta x,y),t)=p((x,y)\rightarrow (x,y\pm\Delta y),t)=\frac{1}{4}$.  Assume that $D=\frac{\Delta x^2}{\Delta t}$ as $\Delta t \rightarrow 0^+>0$.  Derive the PDE for $u(x,y,t)$.  Hint: use Taylor expansion for multivariate functions.

\item Do the same as above for $\Delta x=2\Delta y$.

\item (only for motivated students) Let us assume that each particle can also move across the diagonal such that $p((x,y)\rightarrow (x\pm\Delta x,y),t)=p((x,y)\rightarrow (x,y\pm\Delta y),t)=\alpha$ and $p((x,y)\rightarrow (x\pm\Delta x,y\pm\Delta y),t)=1/4-\alpha$, $\alpha\in(0,1/4)$

\item  Let us now consider a bar or rod in $\mathbb R^3$ which is inhomogeneous in $x$--direction but homogeneous in both $y$ and $z$ directions.  We assume that the density $\rho(x)$ ($gram/cm.^3$), the cross-sectional area $A(x)$ ($cm^2$), the specific heat capacity $c(x)$ ($calorie/gram.degree $) as well as the thermal conductivity $\kappa(x)$ ($calorie/cm. degree. second$) are now functions of $x$ but are constant throughout any particular cross-section.  Therefore, the temperature $u(x,t)$ is a constant for any particular cross-section.  Suppose that there is no heat or cooling resource within the bar.  Show that the equation of such heat flow is
\begin{equation}\label{1}
c(x)\rho(x)A(x)\frac{\partial u}{\partial t}=\kappa(x)A(x)\frac{\partial^2 u}{\partial x^2}+\Big(\kappa(x)A'(x)+A(x)\kappa'(x) \Big)\frac{\partial u}{\partial x},
\end{equation}
where $'$ denotes the derivative taken concerning $x$.  You need to present your justifications in a logically well-ordered way.  Remark: now the arbitrary domain $\Omega$ is a cylinder and its outer normal can be explicitly calculated.

\item A steady-state of a time-dependent equation means that the temperature at any point is independent of time.  Find the steady state of a heat equation over $\Omega=(0, L)$ with the left end temperature fixed at $u_0 $ and the right end fixed at $u_1$.   Remark: you might help you `see' what the flux rate is proportional to the `gradient'.

\item Denote $u_n(x,t):=e^{-D(\frac{n\pi}{L})^2t} \sin \frac{n\pi x}{L}$.

i) show that for each $N<\infty$, the series $\sum_{n=1}^N c_nu_n(x,t)$ is a solution to the following baby heat equation
\[\frac{\partial u}{\partial t}=D\frac{\partial^2 u}{\partial x^2},\]
where $c_n$ are constants.  Remark: the series is a solution for $N=\infty$ as long it converges and we will discuss it later in this course;

ii) assume that $L=\pi$ and $D=0.05$.  Use MATLAB or other software to plot $u_1(x,t)$ over $x\in(0,\pi)$ with $t=0$, $t=0.5$, $t=2$ in the same coordinate.  Do the same for $D=0.1$ and $D=1$.  What are your observations and explain them intuitively;

iii)  again, for $D=1$, use MATLAB or other software to plot $1u_1(x,t)+0.5u_2(x,t)$ over $x\in(0,\pi)$ with $t=0$, $t=0.5$, $t=2$ in the same coordinate;

\item  Let us consider the following generalized heat equation for $m\geq1$
\begin{equation}\label{PME}
u_t= \Delta (u^m),\textbf{x}\in\mathbb R^N, t\in\mathbb R^+,
\end{equation}
which reduces to the classical heat equation when $m=1$, and to Boussinesq's equation when $m=2$.  Note that one can rewrite $u_t=\nabla \cdot (mu^{m-1} \nabla u)$, hence the diffusion rate is recognized as $mu^{m-1}$.  This equation was proposed in the study of ideal gas flowing isentropically in a homogeneous medium or the flow of fluid through porous media (such as oil through the soil), where the law, instead of Fourier's law of constant diffusivity, $ \textbf{J}=- u^{m-1}\nabla u$ is usually observed.  You do not need to know why this particular form is chosen, but it is not surprising to imagine that the flow of dye in water behaves differently from that of the oil in the soil.

A fundamental solution to the problem in 1D was obtained in the 1950s by Russian mathematician Barenblatt, where in $\mathbb R^N$, $N\geq1$, one has
\begin{equation}\label{Barenblatt}
u(\textbf{x},t)=t^{-\alpha}\Big(C-\kappa |\textbf{x}|^2t^{-2\beta}\Big)_+^\frac{1}{m-1},
\end{equation}
where
\[(f)_+:=\max\{f,0\}, \alpha:=\frac{N}{(m-1)N+2}, \beta:=\frac{\alpha}{N}, \kappa =\frac{\alpha(m-1)}{2mN}\]
and $C$ is any positive constant.

(1) Prove that (\ref{Barenblatt}) satisfies the PME (\ref{PME});

(2)  In 1D ($n=1$), choose $m=2$ and $C=10$, and then plot $u(x,t)$ for $t=1$, $2$, $5$ and $10$.


\item  Go to review on the following topics: gradient; directional derivative; multivariate integral; surface integral; divergence theorem; No need to turn in your review. 

\end{enumerate}


\end{document}
\endinput
