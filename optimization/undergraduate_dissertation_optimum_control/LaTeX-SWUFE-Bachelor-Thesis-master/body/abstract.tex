\begin{abstractzh}
    %\zhlipsum[1]
    随着奥密克戎病毒席卷全球,对国家经济水平、人群健康造成巨大威胁,政府实施对人群之间的互动进行宏观调控是缓解疫情危害的重要武器。控制策略作为政府决定的强制性策略必然会引起大众的讨论。在本文,我们设定了SEIR宏观模型并通过建立最优控制函数来找到平衡经济水平和人群健康的最佳决策。首先,我们通过各类数据网站估计和效用相关的参数,设定各类人群选择不同的消费和工作时间来最大化他们的效用,并将人群的消费和工作时间等决策通过古诺模型模拟人们之间的行为,然后设封控策略为控制系数将求解最大效用的函数转变为最优控制问题。得出的结果表明政府管理者应该尽可能早的实施封控策略,并不要提前结束封控措施,防止疫情反弹,尽管在控制初期会导致较低的经济水平,但是长期来看消费水平更高,经济效应更强,并使人群的感染率维持在较低水平。
    \keywordszh{奥密克戎;SEIR模型;最优控制}
\end{abstractzh}

\begin{abstracten}
    %\lipsum[1]
    With the Omicron virus sweeping across the globe, posing a significant threat to national economic levels and public health, government interventions in the form of macroscopic control over interactions among the population have become crucial weapons in mitigating the hazards of the pandemic.   As a mandatory strategy determined by the government, control measures inevitably spark public discussions.   In this study, we establish a macroscopic SEIR model and employ optimal control functions to identify the optimal decisions for balancing economic well-being and public health.   Initially, we estimate parameters related to utility using various data sources, setting diverse population groups to maximize their utility through decisions regarding consumption and working hours.   We simulate the behavior among individuals using the Generalized Nash Equilibrium model to represent their decision-making processes.   Subsequently, by introducing the control coefficient for lockdown policies, we transform the maximization of utility into an optimal control problem.   The results indicate that government authorities should implement lockdown measures as early as possible and refrain from prematurely lifting these restrictions to prevent a resurgence of the pandemic.   Although such measures may lead to lower economic levels in the initial stages of control, the long-term perspective reveals higher consumption levels, stronger economic effects, and the maintenance of a lower infection rate among the population.
    \keywordsen{Omicron; SEIR model; Optimal control}
\end{abstracten}
