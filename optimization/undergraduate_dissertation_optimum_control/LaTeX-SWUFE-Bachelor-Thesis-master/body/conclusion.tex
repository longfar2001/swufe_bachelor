\section{全文总结与展望}
\subsection{全文总结}
在本文中,我们扩展了SEIR模型来研究经济决策和流行病之间的相互作用,寻找平衡经济水平和人群健康的最优策略。在我们的模型中,各类人群通过相对应的不同消费和工作时间的效用函数最大化自身效用,政府管理人员则通过封锁策略制定控制率来最大化社会整体福利,进行了两步优化。针对消费和工作时间的参数估计,通过多种数据网站查找并通过求解不等式约束问题进行计算,定量描述并比较了多种封控政策的效果。获得的主要成果包括:

通过Matlab工具箱fmincon算法得到随时间变动的封控率,表明封控强度是先上升,中期过后开始逐渐放缓。与不控制相比,最优控制策略可以明显减少奥密克戎造成的医疗挤兑以及重症死亡率,对于经济情况,虽然初期的封锁会导致经济衰退严重,但长期来看,消费强度更高,经济效应更加显著。
同时,我们的结果表明,推迟实施或提早结束封控策略都会对经济水平和感染人数产生负面效用,所以我们认为在疫情初期感染扩散程度较低时尽早控制,以及不能提前结束封控避免疫情反弹。

本文还设定了死亡惩罚系数来反应死亡对社会产生的负面效用,提高死亡惩罚系数使得估计的数据更加接近注重民生福利及防控力度的政府,会产生更严格的封控策略,进而导致感染人数和消费量降低。我们也将疫苗接种纳入到模型当中,感染人数迅速降低,经济恢复到疫情之前,接种疫苗可以彻底结束疫情。
\subsection{后续工作展望}
除此以外,针对封控策略还有许多问题需要进一步的探讨,第二章中提出的SEIR模型中,针对潜伏者E我们把他视为从易感者变为感染者的中介变量,有关潜伏者的传染能力对模型参数和传播链条上的影响考虑不足,只是简单地在控制系数前面加一个影响因子来表明潜伏者的控制力度较低。

由于没有获取到足够的每类人群健康状况的额外信息,我们的封控策略是针对所有人群的控制,没有细分到每类人群。部分政府的防控措施比较完善,通过医学检测的手段可以精确跟进每类人群的实时状态,所以能够分别对不同类别的人群实施最佳的封控措施,有助于降低感染人数并大幅提高易感人群的消费和工作时间。在未来研究中,我们可以考虑增加额外的医疗检测系数,以及对易感者、潜伏者、感染者不同的控制系数来制定智能封控策略。
