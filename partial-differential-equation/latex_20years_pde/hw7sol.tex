\NeedsTeXFormat{LaTeX2e}% LaTeX 2.09 can't be used (nor non-LaTeX)
[1994/12/01]% LaTeX date must December 1994 or later
\documentclass[6pt]{article}
\pagestyle{headings}
\setlength{\textwidth}{18cm}
\setlength{\topmargin}{0in}
\setlength{\headsep}{0in}

\title{Introduction to PDEs, Fall 2020}
\author{\textbf{Homework 7, Solutions}}
\date{}

\voffset -2cm \hoffset -1.5cm \textwidth 16cm \textheight 24cm
\renewcommand{\theequation}{\thesection.\arabic{equation}}
\renewcommand{\thefootnote}{\fnsymbol{footnote}}
\usepackage{amsmath}
\usepackage{amsthm}
 \usepackage{textcomp}
\usepackage{esint}
  \usepackage{paralist}
  \usepackage{graphics} %% add this and next lines if pictures should be in esp format
  \usepackage{epsfig} %For pictures: screened artwork should be set up with an 85 or 100 line screen
\usepackage{graphicx}
\usepackage{caption}
\usepackage{subcaption}
\usepackage{epstopdf}%This is to transfer .eps figure to .pdf figure; please compile your paper using PDFLeTex or PDFTeXify.
 \usepackage[colorlinks=true]{hyperref}
 \usepackage{multirow}
\input{amssym.tex}
\def\N{{\Bbb N}}
\def\Z{{\Bbb Z}}
\def\Q{{\Bbb Q}}
\def\R{{\Bbb R}}
\def\C{{\Bbb C}}
\def\SS{{\Bbb S}}

\newtheorem{theorem}{Theorem}[section]
\newtheorem{corollary}{Corollary}
%\newtheorem*{main}{Main Theorem}
\newtheorem{lemma}[theorem]{Lemma}
\newtheorem{proposition}{Proposition}
\newtheorem{conjecture}{Conjecture}
\newtheorem{solution}{Solution}
%\newtheorem{proof}{Proof}
 \numberwithin{equation}{section}
%\newtheorem*{problem}{Problem}
%\theoremstyle{definition}
%\newtheorem{definition}[theorem]{Definition}
\newtheorem{remark}{Remark}
%\newtheorem*{notation}{Notation}
\newcommand{\ep}{\varepsilon}
\newcommand{\eps}[1]{{#1}_{\varepsilon}}
\newcommand{\keywords}


\def\bb{\begin}
\def\bc{\begin{center}}       \def\ec{\end{center}}
\def\ba{\begin{array}}        \def\ea{\end{array}}
\def\be{\begin{equation}}     \def\ee{\end{equation}}
\def\bea{\begin{eqnarray}}    \def\eea{\end{eqnarray}}
\def\beaa{\begin{eqnarray*}}  \def\eeaa{\end{eqnarray*}}
\def\hh{\!\!\!\!}             \def\EM{\hh &   &\hh}
\def\EQ{\hh & = & \hh}        \def\EE{\hh & \equiv & \hh}
\def\LE{\hh & \le & \hh}      \def\GE{\hh & \ge & \hh}
\def\LT{\hh & < & \hh}        \def\GT{\hh & > & \hh}
\def\NE{\hh & \ne & \hh}      \def\AND#1{\hh & #1 & \hh}

\def\r{\right}
\def\lf{\left}
\def\hs{\hspace{0.5cm}}
\def\dint{\displaystyle\int}
\def\dlim{\displaystyle\lim}
\def\dsup{\displaystyle\sup}
\def\dmin{\displaystyle\min}
\def\dmax{\displaystyle\max}
\def\dinf{\displaystyle\inf}

\def\al{\alpha}               \def\bt{\beta}
\def\ep{\varepsilon}
\def\la{\lambda}              \def\vp{\varphi}
\def\da{\delta}               \def\th{\theta}
\def\vth{\vartheta}           \def\nn{\nonumber}
\def\oo{\infty}
\def\dd{\cdots}               \def\pa{\partial}
\def\q{\quad}                 \def\qq{\qquad}
\def\dx{{\dot x}}             \def\ddx{{\ddot x}}
\def\f{\frac}                 \def\fa{\forall\,}
\def\z{\left}                 \def\y{\right}
\def\w{\omega}                \def\bs{\backslash}
\def\ga{\gamma}               \def\si{\sigma}
\def\iint{\int\!\!\!\!\int}
\def\dfrac#1#2{\frac{\displaystyle {#1}}{\displaystyle {#2}}}
\def\mathbb{\Bbb}
\def\bl{\Bigl}
\def\br{\Bigr}
\def\Real{\R}
\def\Proof{\noindent{\bf Proof}\quad}
\def\qed{\hfill$\square$\smallskip}

\begin{document}
\maketitle

\textbf{Name}:\rule{1 in}{0.001 in} \\
\begin{enumerate}
\item In practice, we are motivated to approximate a delta function by some concrete functions \emph{on earth}, and these functions are called the \emph{nascent delta functions}.  (The word \emph{nascent} means \emph{beginning}, or \emph{newly developing}).  One of the natural choices, as we showed in class, is
  \begin{equation}
\delta_\epsilon(x)=\left\{
\begin{array}{ll}
(\epsilon-x)/\epsilon^2,&x\in(0,\epsilon),\\
(x+\epsilon)/\epsilon^2,&x\in(-\epsilon,0),\\
0,&\vert x \vert \geq \epsilon.
\end{array}
\right.
\end{equation}
This hat function seems good: it is intuitively very simple, and it is compactly supported, (i.e., the region $\eta_\epsilon(x)\neq0$ is of the form $(-\epsilon,\epsilon)$, though not compact itself, and is compactly supported in $\mathbb R$).  However, it is not differentiable at the end $x=\pm\epsilon$ which circumvents it from further applications whenever differentiation would be taken.

Give an alternative example of such nascent delta function which is not only compactly supported but also $C^\infty$.  Then show that this new $\delta_\epsilon(x)$ converges to the delta function $\delta(x)$ in distribution.  Note that the heat kernel sounds like a candidate but it is not compactly supported if you understand what it means.
\begin{solution}
skipped
\end{solution}

\item  Show the following facts for $\delta(x)$:

(1).  $\delta(x)=\delta(-x)$;

(2). $\delta(kx)=\frac{\delta(x)}{\vert k\vert }$, where $k$ is a non-zero constant;

(3).  $\int_\mathbb{R} f(x)\delta(x-x_0) dx=f(x_0)$; $\delta(x-x_0)$ is occasionally written as $\delta_{x_0}(x)$;

(4).  Let $f(x)$ be continuous except for a jump--discontinuity at $0$.  Show that
\[\frac{f(0^-)+f(0^+)}{2}=\int_{-\infty}^\infty f(x)\delta(x)dx\]

Remark: For (1) and (2), you are indeed asked to show that they equal in the distribution sense, but not pointwisely.

\begin{solution}
(1). I would like to mention that, whenever to check identities involving $\delta(x)$, it is necessary to check that it holds both pointwisely, as well as in the distribution sense.  First of all, it is easy to check that $\delta(x)=\delta(-x)$ pointwisely (which is only a formal identity), hence I skip it here.  To show that they are the same in the distribution sense, we choose an arbitrary continuous function $\phi(x)$ and have that
\[\int_{-\infty}^\infty \phi(x)\delta(x)dx=\phi(0)=\phi(-0)=\int_{-\infty}^\infty\phi(-x)\delta(x)dx=\int_{-\infty}^\infty\phi(\xi)\delta(-\xi)d\xi,\]
 where we denote $\xi:=-x$ in the last identity.  Therefore we have the desired identity.  The verification of (2) and (3) follows the same approach and I skip it here.


(4).  There are several ways that one can obtain this result and here is one of them.  Let us introduce
\[g(x)=\frac{f(x)+f(-x)}{2},\forall x\in\mathbb R;\]
if we further define
\[g(0):=\frac{f(0^+)+f(0^-)}{2},\]
then it is easy to see that $g(x)$ is continuous over $\mathbb R^n$.  Therefore, by the definition of a dirac--delta function, we have that
\[\int_{\mathbb R}g(x)\delta (x)dx=g(0)=\frac{f(0^+)+f(0^-)}{2};\]
on the other hand, we have that
\begin{align*}
   &\int_{\mathbb R}g(x)\delta (x)dx=\frac{1}{2}\int_{\mathbb R} (f(x)+f(-x))\delta (x)dx  \\
   =& \frac{1}{2}\int_{\mathbb R}f(x)\delta (x)dx+\frac{1}{2}\int_{\mathbb R}f(-x)\delta(x)dx\\
   =&\int_{\mathbb R}f(x)\delta(x)dx,
\end{align*}
where the last identity follows from a change of variable as above.  Another approach of the same spirit is to introduce $F(x)=2g(x)$ and define $F(0)$ as it should be.   This should give a method which is intuitively not straightforward, but rigourous.

\end{solution}


\item  We have defined the dirac-delta function $\delta(x)$ in 1D in class.

(1) generalize the definition to $N$D, $N\geq2$.

(2) give an example of the nascent delta function as above;

\begin{solution}
(1) all requirements in the definition remain the same except that $\mathbb R$ becomes $\mathbb R^N$;

(2) one can simply extend the hat function to $N$D
\end{solution}


\item One can following the same approximation approach to, at least formally, solving the Schr\"odinger's equation describing evolution of the quantum state of a quantum system, discovered and formulated by the Austrian physicist Erwin Schr\"dinger in 1920s.  Solve its one-dimensional version
    \[
\left\{
\begin{array}{ll}
iu_{t}+Du_{xx}=0,&x\in \mathbb R,t>0,\\
u(x,0)=f(x),&x\in \mathbb R,\\
\end{array}
\right.
\]
where $D$ is related to the co--called Planck constant, $i$ is the complex unit such that $i^2=-1$, $u$ and $f$ are complex functions. ($u$ denotes a wave function).
\begin{solution}
skipped
\end{solution}


\item (i) In various problems, we have used the fact: prove that as $M\rightarrow \infty$
\[I_1=\int_{-\infty}^{-M}e^{-\eta ^2}dx\rightarrow 0;\]

(ii)  Prove or disapprove the following: (a)
suppose that $f(x)$ is absolutely integrable, i.e.,
\[\int_\mathbb R|f(x)|dx<\infty\]
then as $M\rightarrow \infty$
\[I=\int_{-\infty}^{-M} |f(x)|dx\rightarrow 0.\]
(b).  Suppose that $f(x)$ is absolutely integrable, then
\[\lim_{|x|\rightarrow\infty} |f(x)|=0.\]
\begin{solution}
skipped
\end{solution}
\item Write down Lebesgue's dominated convergence theorem.
\begin{solution}
skipped
\end{solution}
\end{enumerate}


\end{document}
\endinput
