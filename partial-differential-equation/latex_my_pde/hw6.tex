\NeedsTeXFormat{LaTeX2e}% LaTeX 2.09 can't be used (nor non-LaTeX)
[1994/12/01]% LaTeX date must December 1994 or later
\documentclass[6pt]{article}
\pagestyle{headings}
\setlength{\textwidth}{18cm}
\setlength{\topmargin}{0in}
\setlength{\headsep}{0in}

\title{Introduction to PDEs, Fall 2022}
\author{\textbf{Homework 6} Due Nov 14}
\date{}

\voffset -1.25cm \hoffset -3.5cm \textwidth 18cm \textheight 25cm
\renewcommand{\theequation}{\thesection.\arabic{equation}}
\renewcommand{\thefootnote}{\fnsymbol{footnote}}
\usepackage{amsmath}
\usepackage{amsthm}
%\usepackage{esint}
  \usepackage{paralist}
  \usepackage{graphics} %% add this and next lines if pictures should be in esp format
  \usepackage{epsfig} %For pictures: screened artwork should be set up with an 85 or 100 line screen
\usepackage{graphicx}
\usepackage{caption}
\usepackage{subcaption}
\usepackage{epstopdf}%This is to transfer .eps figure to .pdf figure; please compile your paper using PDFLeTex or PDFTeXify.
 \usepackage[colorlinks=true]{hyperref}
 \usepackage{multirow}
\input{amssym.tex}
\def\N{{\Bbb N}}
\def\Z{{\Bbb Z}}
\def\Q{{\Bbb Q}}
\def\R{{\Bbb R}}
\def\C{{\Bbb C}}
\def\SS{{\Bbb S}}

\newtheorem{theorem}{Theorem}[section]
\newtheorem{corollary}{Corollary}
%\newtheorem*{main}{Main Theorem}
\newtheorem{lemma}[theorem]{Lemma}
\newtheorem{proposition}{Proposition}
\newtheorem{conjecture}{Conjecture}
\newtheorem{solution}{Solution}
%\newtheorem{proof}{Proof}
 \numberwithin{equation}{section}
%\newtheorem*{problem}{Problem}
%\theoremstyle{definition}
%\newtheorem{definition}[theorem]{Definition}
\newtheorem{remark}{Remark}
%\newtheorem*{notation}{Notation}
\newcommand{\ep}{\varepsilon}
\newcommand{\eps}[1]{{#1}_{\varepsilon}}
\newcommand{\keywords}

\def\bb{\begin}
\def\bc{\begin{center}}       \def\ec{\end{center}}
\def\ba{\begin{array}}        \def\ea{\end{array}}
\def\be{\begin{equation}}     \def\ee{\end{equation}}
\def\bea{\begin{eqnarray}}    \def\eea{\end{eqnarray}}
\def\beaa{\begin{eqnarray*}}  \def\eeaa{\end{eqnarray*}}
\def\hh{\!\!\!\!}             \def\EM{\hh &   &\hh}
\def\EQ{\hh & = & \hh}        \def\EE{\hh & \equiv & \hh}
\def\LE{\hh & \le & \hh}      \def\GE{\hh & \ge & \hh}
\def\LT{\hh & < & \hh}        \def\GT{\hh & > & \hh}
\def\NE{\hh & \ne & \hh}      \def\AND#1{\hh & #1 & \hh}

\def\r{\right}
\def\lf{\left}
\def\hs{\hspace{0.5cm}}
\def\dint{\displaystyle\int}
\def\dlim{\displaystyle\lim}
\def\dsup{\displaystyle\sup}
\def\dmin{\displaystyle\min}
\def\dmax{\displaystyle\max}
\def\dinf{\displaystyle\inf}

\def\al{\alpha}               \def\bt{\beta}
\def\ep{\varepsilon}
\def\la{\lambda}              \def\vp{\varphi}
\def\da{\delta}               \def\th{\theta}
\def\vth{\vartheta}           \def\nn{\nonumber}
\def\oo{\infty}
\def\dd{\cdots}               \def\pa{\partial}
\def\q{\quad}                 \def\qq{\qquad}
\def\dx{{\dot x}}             \def\ddx{{\ddot x}}
\def\f{\frac}                 \def\fa{\forall\,}
\def\z{\left}                 \def\y{\right}
\def\w{\omega}                \def\bs{\backslash}
\def\ga{\gamma}               \def\si{\sigma}
\def\iint{\int\!\!\!\!\int}
\def\dfrac#1#2{\frac{\displaystyle {#1}}{\displaystyle {#2}}}
\def\mathbb{\Bbb}
\def\bl{\Bigl}
\def\br{\Bigr}
\def\Real{\R}
\def\Proof{\noindent{\bf Proof}\quad}
\def\qed{\hfill$\square$\smallskip}

\begin{document}
\maketitle

\textbf{Name}:\rule{1 in}{0.001 in} \\


\begin{enumerate}
\item In measure theory, there are two additional convergence manners for $f_n\rightarrow f$: convergence in measure (called convergence in probability) and convergence almost everywhere (convergence almost surely).  This should give you a flavor that probability is a measure and vice versa.  Convergence in measure states that for each fixed $\varepsilon>0$
    \[m\big(\{x\in\Omega: |f_n(x)-f(x)|\geq\varepsilon\}\big)\rightarrow 0, \text{~as~}n\rightarrow \infty,\]
    and convergence almost everywhere means that the measure of the non-convergence region is zero, i.e., for each fixed $\varepsilon>0$
    \[m\big(\{x\in\Omega: \lim_{n\rightarrow\infty}|f_n(x)-f(x)|\geq\varepsilon\}\big)=0.\]
i) what are the relationships between these two convergence manners?  Prove your claims or give a counter-example.

ii) what are their relationships between strong convergence (convergence in $L^2$ for instance)?  Prove your claims or give a counter-example.

I would like to point out that convergence is global behavior in strong contrast to pointwise convergence since all the points are involved in the convergence limit.

\item  It is known that strong convergence implies weak convergence, while not the converse.  One counter-example we mentioned in class is $f_n(x):=\sin nx$ over $(0,2\pi)$.

(i)  Prove that $\sin nx \nrightarrow 0 $ in $L^2((0,2\pi))$.

(ii)  Prove that
$\sin nx \rightharpoonup 0$ weakly by showing
\[\int_0^{2\pi} g(x)\sin nx dx\rightarrow 0=\Big(\int_0^{2\pi} g(x)0  dx\Big),\forall g\in L^2((0,2\pi)).\]
If suffices even if $g\in L^1$.  Hint: Riemann--Lebesgue lemma.

\item  We recall that $f_n(x) \rightharpoonup f(x)$ weakly in $L^p$ (resp. convergence in distribution) if for any $\phi\in L^q$ (resp. continuous and bounded), which is its conjugate space with $\frac{1}{p}+\frac{1}{q}=1$, we have that
\[\int_\Omega f_n\phi dx\rightarrow \int_\Omega f\phi dx.\]
Here we see that for any $g$ in $L^q$
\[<\cdot, g>=\int_\Omega \cdot g\]
defines a bounded linear functional for $L^p$.  Then we also call $L^q$ \emph{the dual space} of $L^p$ since any element in $L^q$ defines a functional for $L^q$.

(i)  Another type of convergence that you may see sometimes is $\Vert f_n \Vert_p\rightarrow \Vert f\Vert_p$, which merely states the convergence of a sequence of real numbers.  Prove that if $f_n \rightarrow f$ in $L^p$, then $\Vert f_n \Vert_p\rightarrow \Vert f\Vert_p$ (Use Minkowski triangle inequality); however the opposite statement is not necessarily true.  Give a counter-example and show it;

(ii) We have proved that strong convergence in $L^p$ implies the weak convergence by Holder's inequality, however, the opposite statement is not necessarily true.  For example, prove that $\sin nx$ converges to zero weakly, but not strongly in $L^p$.  Hint: Riemann--Lebesgue lemma;

(iii) Prove that, if $f_n \rightharpoonup f$ weakly and $\Vert f_n \Vert_p\rightarrow \Vert f\Vert_p$, then $f_n\rightarrow f$ strongly.
\end{enumerate}


\end{document}
\endinput




