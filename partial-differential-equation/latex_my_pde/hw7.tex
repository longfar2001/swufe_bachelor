\NeedsTeXFormat{LaTeX2e}% LaTeX 2.09 can't be used (nor non-LaTeX)
[1994/12/01]% LaTeX date must December 1994 or later
\documentclass[6pt]{article}
\pagestyle{headings}
\setlength{\textwidth}{18cm}
\setlength{\topmargin}{0in}
\setlength{\headsep}{0in}

\title{Introduction to PDEs, Fall 2022}
\author{\textbf{Homework 7, Due Nov 24}}
\date{}

\voffset -2cm \hoffset -1.5cm \textwidth 16cm \textheight 24cm
\renewcommand{\theequation}{\thesection.\arabic{equation}}
\renewcommand{\thefootnote}{\fnsymbol{footnote}}
\usepackage{amsmath}
\usepackage{amsthm}
 \usepackage{textcomp}
\usepackage{esint}
  \usepackage{paralist}
  \usepackage{graphics} %% add this and next lines if pictures should be in esp format
  \usepackage{epsfig} %For pictures: screened artwork should be set up with an 85 or 100 line screen
\usepackage{graphicx}
\usepackage{caption}
\usepackage{subcaption}
\usepackage{epstopdf}%This is to transfer .eps figure to .pdf figure; please compile your paper using PDFLeTex or PDFTeXify.
 \usepackage[colorlinks=true]{hyperref}
 \usepackage{multirow}
\input{amssym.tex}
\def\N{{\Bbb N}}
\def\Z{{\Bbb Z}}
\def\Q{{\Bbb Q}}
\def\R{{\Bbb R}}
\def\C{{\Bbb C}}
\def\SS{{\Bbb S}}

\newtheorem{theorem}{Theorem}[section]
\newtheorem{corollary}{Corollary}
%\newtheorem*{main}{Main Theorem}
\newtheorem{lemma}[theorem]{Lemma}
\newtheorem{proposition}{Proposition}
\newtheorem{conjecture}{Conjecture}
\newtheorem{solution}{Solution}
%\newtheorem{proof}{Proof}
 \numberwithin{equation}{section}
%\newtheorem*{problem}{Problem}
%\theoremstyle{definition}
%\newtheorem{definition}[theorem]{Definition}
\newtheorem{remark}{Remark}
%\newtheorem*{notation}{Notation}
\newcommand{\ep}{\varepsilon}
\newcommand{\eps}[1]{{#1}_{\varepsilon}}
\newcommand{\keywords}


\def\bb{\begin}
\def\bc{\begin{center}}       \def\ec{\end{center}}
\def\ba{\begin{array}}        \def\ea{\end{array}}
\def\be{\begin{equation}}     \def\ee{\end{equation}}
\def\bea{\begin{eqnarray}}    \def\eea{\end{eqnarray}}
\def\beaa{\begin{eqnarray*}}  \def\eeaa{\end{eqnarray*}}
\def\hh{\!\!\!\!}             \def\EM{\hh &   &\hh}
\def\EQ{\hh & = & \hh}        \def\EE{\hh & \equiv & \hh}
\def\LE{\hh & \le & \hh}      \def\GE{\hh & \ge & \hh}
\def\LT{\hh & < & \hh}        \def\GT{\hh & > & \hh}
\def\NE{\hh & \ne & \hh}      \def\AND#1{\hh & #1 & \hh}

\def\r{\right}
\def\lf{\left}
\def\hs{\hspace{0.5cm}}
\def\dint{\displaystyle\int}
\def\dlim{\displaystyle\lim}
\def\dsup{\displaystyle\sup}
\def\dmin{\displaystyle\min}
\def\dmax{\displaystyle\max}
\def\dinf{\displaystyle\inf}

\def\al{\alpha}               \def\bt{\beta}
\def\ep{\varepsilon}
\def\la{\lambda}              \def\vp{\varphi}
\def\da{\delta}               \def\th{\theta}
\def\vth{\vartheta}           \def\nn{\nonumber}
\def\oo{\infty}
\def\dd{\cdots}               \def\pa{\partial}
\def\q{\quad}                 \def\qq{\qquad}
\def\dx{{\dot x}}             \def\ddx{{\ddot x}}
\def\f{\frac}                 \def\fa{\forall\,}
\def\z{\left}                 \def\y{\right}
\def\w{\omega}                \def\bs{\backslash}
\def\ga{\gamma}               \def\si{\sigma}
\def\iint{\int\!\!\!\!\int}
\def\dfrac#1#2{\frac{\displaystyle {#1}}{\displaystyle {#2}}}
\def\mathbb{\Bbb}
\def\bl{\Bigl}
\def\br{\Bigr}
\def\Real{\R}
\def\Proof{\noindent{\bf Proof}\quad}
\def\qed{\hfill$\square$\smallskip}

\begin{document}
\maketitle

\textbf{Name}:\rule{1 in}{0.001 in} \\
\begin{enumerate}

\item In class, we arrived at an integral of the following form when evaluating $G^{\pm}_L$
\[I(c)=\int_0^\infty e^{-w^2b} \cos (wc) dw,\]
where $b$ and $c$ are constants.

(i) (Optional) Evaluate this integral through integration by parts or any method you know;

(ii).  An alternative approach is to solve an ODE that $I(c)$ satisfies.  Show that $I(c)$ satisfies
\[\frac{dI(c)}{dc}=-\frac{c}{2b}I(c);\]

(iii).  Show that $I(0)=\sqrt \frac{\pi}{4b}$ and solve the ODE in (ii) to find $I(c)$.

\item We know from class that the solution to the following problem
\begin{equation}\label{halfinfty}
\left\{
\begin{array}{ll}
u_t=D u_{xx},& x\in (0,\infty), t\in\mathbb R^+,\\
u(x,0)=\phi(x),&x \in (0,\infty),\\
u(0,t)=u(\infty,t)=0,&t\in\mathbb R^+.
\end{array}
\right.
\end{equation}
is given in the following form\footnote{Throughout this homework, and probably the whole course, $G(\xi;x,t)$ is the heat kernel and it is explicitly given by
\[G(\xi;x,t)=\frac{1}{\sqrt{4\pi Dt}}e^{-\frac{|x-\xi|^2}{4Dt}}.\]}
\[u(x,t)=\int_0^\infty \big(G^-(\xi;x,t)-G^+(\xi;x,t)\big)\phi(\xi)d\xi.\]
Note that this integral above can be evaluated symbolically.  Choose $D=1$ and the initial data to be $\phi(x)\equiv 1$ for $x\in(0,1)\cup(2,3)$ and $\phi(x)\equiv0$ otherwise.  Plot the solution of (\ref{halfinfty}) at times $t=10^{-4},10^{-3},0.1,0.5$,1 and 5.  Note that this integral over $(0,\infty)$ must be truncated over $(0,L)$ for $L$ large.  Choose your own $L$.  (You should know how to choose such $L$ up to certain accuracy by now).


\item  Let us consider the following IBVP over half line $(0,\infty)$ with Neumann boundary condition
\begin{equation}\label{NBC}
\left\{
\begin{array}{ll}
u_t=D u_{xx},& x\in (0,\infty), t\in\mathbb R^+,\\
u(x,0)=\phi(x),&x \in (0,\infty),\\
u_x(0,t)=0,&t\in(0,\infty),
\end{array}
\right.
\end{equation}
Similar as in class, tackle this problem by first solving its counterpart in $(0,L)$ and then sending $L\rightarrow\infty$.  Hint: the suggested solution is
\[u(x,t)=\int_0^\infty \Big(G(\xi;x,t)+G(x,t;-\xi)\Big)\phi(\xi)d\xi.\]

\item  Let us consider the following Cauchy problem
\begin{equation}\label{infty}
\left\{
\begin{array}{ll}
u_t=D u_{xx},& x\in (-\infty,\infty), t\in\mathbb R^+,\\
u(x,0)=\phi(x),&x \in (-\infty,\infty).
\end{array}
\right.
\end{equation}
We can approximate the solution to this problem by first solving its counterpart in $(-L,L)$, which has been in a previous homework, and then sending $L\rightarrow\infty$.

Consider
\begin{equation}\label{L}
\left\{
\begin{array}{ll}
u_t=D u_{xx},& x\in (-L,L), t\in\mathbb R^+,\\
u(x,0)=\phi(x),&x \in (-L,L),\\
u(-L,t)=u(L,t)=0,& t\in\mathbb R^+.\\
\end{array}
\right.
\end{equation}

(i).  write the solution to (\ref{L}) in terms of infinite series; you just present your final results, no need to show the details here;

(ii).  write the series above into an integral and then evaluate this integral by sending $L\rightarrow \infty$.  Suggested answer:
\begin{equation}\label{sol}
u(x,t)=\int_{\mathbb R}G(\xi;x,t)\phi(\xi)d\xi,
\end{equation}

We shall see several important applications of solution (\ref{sol}) in the future.

\item   The heat kernel $G(\xi;x,t)$ is sometimes called fundamental solution of heat equation
\[G(\xi;x,t)=\frac{1}{\sqrt{4\pi Dt}}e^{-\frac{(x-\xi)^2}{4Dt}}.\]
Prove that

(i) $|\frac{\partial G}{\partial x}|\rightarrow 0$ as $|x|\rightarrow\infty$ for each $t$ and $\xi$.  Prove the same for $\frac{\partial^m G}{\partial x^m}$ for each $m\in \mathbb N^+$;

(ii) $G_t=DG_{xx}$, $x\in\mathbb R,t\in\mathbb R^+$;

(iii) $\int_{\mathbb R}G(\xi;x,t)dx=1$.

Remark:  I would like to note that we write the kernel $G(\xi;x,t)$ and $G(\xi;x,t)$ interchangeably.  The former is to highlight the eventual solution as a function of $x$ and $t$, whereas the latter is to focus on treating $\xi$ as the integration variable whenever applicable.

\item To give yourself some physical intuitions on the heat kernel, let us consider the following situation in $\mathbb R$: put two separate unit thermal heat at locations $\xi=-1$ and $\xi=1$ respectively at time $t=0$.  Suppose that the temperature $u(x,t)$ satisfies the heat equation with diffusion rate $D=1$, then it is given by the following explicit form
    \[u(x,t)=G(x,t;-1)+G(x,t;1)=\frac{1}{\sqrt{4\pi t}}\Big(e^{-\frac{(x+1)^2}{4t}}+e^{-\frac{(x-1)^2}{4t}}\Big).\]
Plot $u(x,t)$ over $x\in(-5,5)$ with $t=0.01, 0.02, 0.05$, $0.1$ and $1$ \emph{on the same coordinate} in $(-R,R)$ (if $R$ is large, then it approximates the whole line)  to illustrate your results--please use different colors and/or line styles for better effects.  We will know more about the physical intuition in the future; indeed you should already have an intuition about: i) the evolution of the thermal energy; ii) the connect between diffusion and Brownian motion or normal distribution.)

\item

Consider the following problem
\begin{equation}\label{advection}
\left\{
\begin{array}{ll}
u_t=D u_{xx}-\alpha u_x-ru,& x\in (-\infty,0), t\in\mathbb R^+,\\
u(x,0)=\phi(x)\geq,\neq0,&x \in (\infty,0),\\
u(-\infty,t)=e^{-rt}K>0,u(0,t)=0,& t\in\mathbb R^+,\\
\end{array}
\right.
\end{equation}
where $D$, $\alpha$, $r$ and $K$ are positive constants.

Let visit it truncated problem
\begin{equation}\label{advectionL}
\left\{
\begin{array}{ll}
u_t=D u_{xx}-\alpha u_x-ru,& x\in (-L,0), t\in\mathbb R^+,\\
u(x,0)=\phi(x)\geq,\neq0,&x \in (\infty,0),\\
u(-L,t)=e^{-rt}K>0,u(0,t)=0,& t\in\mathbb R^+.\\
\end{array}
\right.
\end{equation}


(i) Solve \eqref{advectionL} in terms of infinite series.  Hint: its boundary condition is inhomogeneous;

(ii) Send $L$ to infinity and then find the limiting solution in terms of an integral.



\end{enumerate}


\end{document}
\endinput
